\documentclass{article}

%Algorithm library
\usepackage{algorithm}
\usepackage{algpseudocode}

%Math library
\usepackage{amsmath} % for math
\usepackage{amsthm} % for theorems and proofs
\usepackage{amssymb}


% Definition for theorems, lemma
\newtheorem{thm}{Theorem}[section]
 \theoremstyle{definition}
 \newtheorem{dfn}{Definition}[section]
 \theoremstyle{remark}
 \newtheorem{note}{Note}[section]
 \theoremstyle{plain}
 \newtheorem{lem}[thm]{Lemma}

\begin{document}

\title{Problem Set 9}
\date{04/28/2017}
\author{Noella James}
\maketitle
collaborators: none\\

\section*{Problem 9-2: Load Balancing}

\subsection*{Part A}

\begin{lem}
$k$ is at most $2m$.
\end{lem}
\begin{proof}
Proof by Contradiction: Let us assume case for when $k > 2m$. We consider the case for where we are about to assign task $i = 2m + 1$ on a machine. The load on each machine $L > \frac{2L*}{3}$. \\
When you add the i'th job, you're adding a job with weight $w_i > \frac{L*}{3}$.\\
Thus, when the job is assigned on a machine, the total load on machine $L > \frac{2L*}{3} +  \frac{L*}{3} > L*$. L* is the maximum load on a machine. Since the maximum load cannot be exceeded, we have a contradiction. This completes the proof. 

\end{proof}

\subsection*{Part B}

The SORTEDBALANCE algorithm sorts the tasks based on the weights in decreasing order. For each task, it is allocated to the least loaded machine. We consider two cases.\\
Case 1:  $0 \leq k \leq m$
Since $k \leq m$, the algorithm allocates a task to each machine. Since we allocate based on sorted order, $L** = w_1$. For any task $i$ where $i\leq k$.
\begin{equation}
w_i \leq L**
\end{equation}
\begin{equation}
L** \leq L
\end{equation}
\begin{equation}
L** > \frac{L*}{3}
\end{equation}

Case 2: $m < k \leq 2m$
When $k > m$, the algorithm allocates more than one task to at least one machine. Additionally, the maximum amount of tasks one can allocate to any machine is 2 tasks. If we allocate more than two tasks, the total weight allocated on the machine exceeds $L*$. Thus we can claim the following:
\begin{equation}
L** > \frac{2 L*}{3}
\end{equation}


\subsection*{Part C}
From the class lectures, we know that the following two claims are true:
\begin{equation}
	w \leq L* 
\end{equation}
\begin{equation}
	L \leq L* 
\end{equation}
When $i \leq k$, we can claim
\begin{equation}
	w_i > \frac{L*}{3}
\end{equation}
For $i > k$ we can claim
\begin{equation}
	w_i \leq \frac{L*}{3}
\end{equation}

Let's assume that all tasks up to $k$ have already been allocated to the machines. We are allocating task $i$ where $i > k$ to a machine. \\
Case 1: $L = 0$\\
\begin{equation}
L+w_i = w
\end{equation}
By equation 8
\begin{equation}
w_i \leq \frac{L*}{3} 
\end{equation}
Case 2: $L > 0$\\
By equations 6 and 10
\begin{equation}
L + w_i \leq L* + \frac{L*}{3}
\end{equation}
\begin{equation}
L* + \frac{L*}{3} = \frac{4L*}{3}
\end{equation}
Thus,
\begin{equation}
L  \leq \frac{4L*}{3}
\end{equation}
\end{document}