\documentclass{article}

%Algorithm library
\usepackage{algorithm}
\usepackage{algpseudocode}

%Math library
\usepackage{amsmath} % for math
\usepackage{amsthm} % for theorems and proofs
\usepackage{amssymb}


% Definition for theorems, lemma
\newtheorem{thm}{Theorem}[section]
 \theoremstyle{definition}
 \newtheorem{dfn}{Definition}[section]
 \theoremstyle{remark}
 \newtheorem{note}{Note}[section]
 \theoremstyle{plain}
 \newtheorem{lem}[thm]{Lemma}

\begin{document}

\title{Problem Set 7}
\date{04/17/2017}
\author{Noella James}
\maketitle
collaborators: none\\

\section*{Problem 7-2: Reductions and Algorithms}

\subsection*{Reduction}

Finding a satisfiable assignment for boolean formula $\phi$ is a SAT problem. We develop a reduction to reduce a SAT formula to a 3SAT formula. $\phi$ could have clauses of various length and we provide reduction for clauses of various length. There are $4$ cases to consider. For the purpose of simplicity, we have used $x_j$, but we could replace it with its negation $x_j'$.  

\subsubsection*{Case 1: $k = 1$} 
Assume there is a clause $C_i$ which has only one variable $x$. We can transform this clause $C_i$ by converting it into $4$ clauses and by introducing $2$ new variables $z_1$ and $z_2$. The clause $C_i$ can now be written as \{$x \vee z_1 \vee z_2$\}, \{$x \vee z_1 \vee z_2'$\}, \{$x \vee z_1' \vee z_2$\}, and \{$x \vee z_1' \vee z_2'$\}. The only way these clauses can simultaneously satisfied is if x is true. This also means that the original clause $C_i$ will be satisfied. 

\subsubsection*{Case 2: $k = 2$} 
Assume there is a clause $C_i$ which has two variables $x_1$ and $x_2$. We create a new variable $z$ and two new clauses \{$x_1 \vee x_2 \vee z$\} and \{$x_1 \vee x_2 \vee z'$\}. The only way to satisfy both of these clauses is for one of $x_1$ or $x_2$ to be true. This also means that the original clause $C_i$ will be satisfied.

\subsubsection*{Case 3: $k = 3$} 
This implies that $C_i$ = $\{x_1 \vee x_2 \vee x_3\}$. Thus, the clause is in 3SAT form and be transferred as is.

\subsubsection*{Case 4: $k \geq 4$}
Let $C_i$ be equal to $\{x_1 \vee x_2 \vee \ldots \vee x_k\}$. We create $k-3$ new variables and $k-2$ new clauses in a chain where for $2 \leq j \leq j-3$, $C_{i,j} = \{z_{i,j-1} \vee x_{j+1} \vee z_{i,j}'\}$, $C_{i,1} = \{x_1 \vee x_2 \vee z_{i,1}'\}$, and $C_{i, k-2} = \{z_{i,k-3} \vee x_{k-1} \vee x_k\}$\\

If none of the original literals in $C_i$ is true, then there are not enough free variables to be able to satisfy all of the new subclasses. If you satisfy  $C_{i,1}$ by setting $z_{i,1}$ to false, it would require $z_{1,2} = $ false and so on until $C_{i,k-2}$ cannot be satisfied and thus the clause cannot be satisfied. However, if any of the single literal $x_i$ is equal to true, then we have $k-3$ free variables and $k-3$ remaining clauses, and thus each of the clauses can be satisfied. Thus, $C_i$ can be satisfied.

\subsection*{Complexity}
Assume that there are n clauses and m total literals, the total complex of the transformation is $O(m+n)$.

\subsection*{Algorithm}
1. Convert a SAT problem to 3SAT as per the reduction steps listed above.\\
2. Pass the 3SAT problem thus reduced to the Black Box that can solve 3SAT problems.\\
3. If the Black Box returns that a satisfiable assignment of variables exists, then return that a satisfiable assignment exists for $\varphi$.\\
4. Else return that a satisfiable assignment for $\varphi$ does not exist.
\end{document}