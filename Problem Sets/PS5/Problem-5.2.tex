\documentclass{article}

%Algorithm library
\usepackage{algorithm}
\usepackage{algpseudocode}

%Math library
\usepackage{amsmath} % for math
\usepackage{amsthm} % for theorems and proofs
\usepackage{amssymb}


% Definition for theorems, lemma
\newtheorem{thm}{Theorem}[section]
 \theoremstyle{definition}
 \newtheorem{dfn}{Definition}[section]
 \theoremstyle{remark}
 \newtheorem{note}{Note}[section]
 \theoremstyle{plain}
 \newtheorem{lem}[thm]{Lemma}

\begin{document}

\title{Problem Set 5}
\date{03/17/2017}
\author{Noella James}
\maketitle
collaborators: none\\

\section*{Problem 5-2: Broken Keys}

\subsection*{Recurrence Equation}

The problem requires a algorithm to transform string1 to string2.  Lets represent string1 as X and string2 as Y.  X can be represent as a set of symbols $(X_1, X_2, \ldots X_m)$. Y can be represent as a set of symbols $(Y_1, Y_2, \ldots Y_n)$.  The problem indicates that the number of presses required to enter a character $x$ is $N[x]$.  The problem does not specify the number of times the delete key has to be pressed to delete a character from string1 (X).  We present the value as $\alpha$. \\\\
In the derivation of the optimal solution there are 4 cases to be considered.\\
\subsubsection*{Case 1}
The last character in string X is not required for string Y and thus should be deleted.  The recurrence relation is\\
$OPT(m,n) = \alpha + OPT(m-1,n)$\\
\subsubsection*{Case 2}
The last character in string Y  need to be entered.  The recurrence relation is\\
$OPT(m,n) = N[Y_n] + OPT(m,n-1)$\\
\subsubsection*{Case 3}
The last character of X and Y are same and thus the mouse pointer can be shifted one place towards left. Thus, the recurrence relationship is\\
$OPT(m,n) = OPT(m-1,n-1)$\\
\subsubsection*{Case 4}
The last character of X and Y are different and thus the last character in X needs to be deleted and the last character Y needs to be entered.  Thus, the recurrence relationship is\\
$OPT(m,n) = \alpha + N[Y_n] + OPT(m-1,n-1)$\\\\
\subsubsection*{Recurrence equation}
The algorithm should chose the minimum value of these 4 options.  Thus, the final recurrence relationship is\\
$OPT(m,n) = min[ \alpha + OPT(m-1,n) ,\quad   N[Y_n] + OPT(m,n-1) ,\quad OPT(m-1,n-1)  ,\quad  \alpha + N[Y_n] + OPT(m-1,n-1) ]$\\\\

\subsection*{Algorithm}

\begin{algorithm}
\caption{MINIMUM KEYSTROKES}\label{Find the minimum number of keystrokes needed to change string1 to string2}
\begin{algorithmic}[1]
\Procedure{MIN-KEYSTROKES}{$X, Y$} \Comment{ $X = \{X_1, X_2,\ldots,X_m\}, Y = \{Y_1, Y_2,\ldots, Y_n\}$, Array $N[x] =$ number of times a character x is pressed for it to be typed, $\alpha = $ number of times delete key has to be pressed to delete a character}
        \State Array $A[0\ldots m, 0 \ldots n]$
        \State Initialize $A[i, 0]  = i\alpha$ for each $i$
        \For{$j = 1 \ldots n$}
         	\State $A[0,j] = A[0,j-1] + N[x_j]$
        \EndFor
	\For{$j = 1\ldots n$}
		\For{$i = 1\ldots m$}
			\State $A[i,j] = min[\alpha + A(i-1,j)  , \quad  N[y_j] + A(i,j-1) , \quad  A(i-1,j-1), \quad \alpha + N[y_j] +  A(i-1,j-1) ]$ \Comment{Recurrence relationship as seen above}
		\EndFor
	\EndFor
	\State \Return $A[m,n]$
\EndProcedure

\end{algorithmic}
\end{algorithm}

\subsection*{Complexity}
String1 = X = $\{X_1, X_2, \ldots, X_m\}$ has $m$ characters. String2 = Y = $\{Y_1, Y_2, \ldots, Y_n\}$ has $n$ characters. The outer for loop runs $n$ times and the inner loop runs $m$ times. The complexity of the recurrence statement is $O(1)$. Therefore, the total complexity is $O(nm)$.

\subsection*{Correctness}
\begin{lem}
The algorithm $MINIMUM KEYSTROKES(X, Y)$ correctly computes the minimum number of keystrokes needed to convert X to Y.
\end{lem}
\begin{proof}
We will prove this by induction on $i + j$.
\subsubsection*{Base Case:}
When $i+j = 0$, we have $i=j=0$, and we don't have to convert any strings. The function returns 0, which is correct since 0 keystrokes are needed to convert X to Y. Indeed $OPT(0,0) = 0.$
\subsubsection*{Induction Hypothesis:}
Now consider arbitrary values of $i$ and $j$, and suppose the statement is true for all pairs $(i',j')$ with $i' + j' < i + j$. The four cases that exist are \\
1. We delete $X_{i}$. \\
2. We type $Y_j$. \\ 
3. $X_{i}$ is the same as $Y_j$ and nothing needs to be done.\\
4. We delete $X_{i}$ and type $Y_j$.\\
We choose the minimum of the four options. Thus,
\begin{equation*}
\begin{split}
 A[i,j] &= min[\alpha + A(i-1,j)  , \quad  N[y_j] + A(i,j-1) , \quad  A(i-1,j-1), \quad \alpha + N[y_j] +  A(i-1,j-1) ]\\
 &= min [ \alpha + OPT(i-1,j),\,  N[y_j] + OPT(i,j-1), \,  OPT(i-1,j-1),\, \alpha + N[y_j] +  OPT(i-1,j-1) ]\\
 &= OPT(i,j)
\end{split} 
\end{equation*}
Thus, this completes the proof.\\
\end{proof}

\end{document}