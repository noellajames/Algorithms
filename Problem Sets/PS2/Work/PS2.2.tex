\documentclass{article}

%Algorithm library
\usepackage{algorithm}
\usepackage{algpseudocode}

%Math library
\usepackage{amsmath} % for math
\usepackage{amsthm} % for theorems and proofs
\usepackage{amssymb}

% Definition for theorems, lemma
\newtheorem{thm}{Theorem}[section]
 \theoremstyle{definition}
 \newtheorem{dfn}{Definition}[section]
 \theoremstyle{remark}
 \newtheorem{note}{Note}[section]
 \theoremstyle{plain}
 \newtheorem{lem}[thm]{Lemma}

\begin{document}

Noella James\\
Collaborators: none\\

\section{Problem 2-2: OCD}

\begin{lem}
Given the previous algorithm, the result will always be the smallest number of containers you can fill $L$ gallons of oil in.
\end{lem}

\begin{proof}
By Induction: A natural number $L$, where $L \ge 1$.\\
\textbf{Base Case:}  $L = 1$\\
Given the algorithm, a 1 gallon container will be used. The base case olds true that the smallest number of containers you can fill $L$ gallons of oil in since only one container is used.\\
\textbf{Induction Hypotheses:} 
Given $k$ gallons of oil, there exists a method to store the $k$ gallons of oil in as few different containers as possible while ensuring that every container you store the oil in is full.\\
\textbf{Induction Step:} 
For $k+1$ gallons of oil, we want to show that there also exists a method to store the $k+1$ gallons of oil in as few different containers as possible while ensuring that every container you store the oil in is full. \\
It is possible to find such an $i$ where $i$ is between $0$ to $1000$ where $2^i  \le k+1$.\\
$Z = k+1 - 2^i$\\
$Z$ is greater than or equal to $0$ and it is less than or equal to $k$.\\
By Induction Hypotheses , there exists a method to store the $Z$ gallons of oil in as few different containers as possible while ensuring that every container you store the oil in is full. Thus, we conclude the lemma is true.
\end{proof}

\begin{lem}
Given the previous algorithm, this algorithm finds the optimal solution.
\end{lem}

\begin{proof}
Assume there is an optimal solution $\theta$. \\
By Induction: A natural number $L$, where $L \le 1$, the solution returned by the previous algorithm is the same as $\theta$.\\
\textbf{Base Case:}  $L = 1$\\
The previous algorithm will therefore return $1$ container of size $1$-gallon.\\
$\theta$ will also return $1$ container of size $1$-gallon since that is the optimal solution.\\
\textbf{Induction Hypotheses:} 
Given $k$ gallons of oil, the previous algorithm and $\theta$ will return the same count and sizes of containers to store the $k$ gallons of oil.\\
\textbf{Induction Step:} 
For $k+1$ gallons of oil, we want to show that the previous algorithm and $\theta$ will return the same count and sizes of containers to store the $k+1$ gallons of oil.\\
It is possible to find such an $i$ where $i$ is between $0$ to $1000$ where $2^i  \le k+1$.\\
$Z = k+1 - 2^i$\\
$Z$ is greater than or equal to $0$ and it is less than or equal to $k$.\\
Both $\theta$ and the previous algorithm will choose the $2^i$ container because that is the maximum container that can be chosen given the rule that holds the most oil and contains no empty space.
Since $Z \ge 0$ and $Z \le k$, both $\theta$ and the previous algorithm will choose the same sequence of containers by the Induction Hypotheses.
Thus, we conclude that the lemma is true.
\end{proof}

\end{document}