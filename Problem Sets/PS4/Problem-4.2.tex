\documentclass{article}

%Algorithm library
\usepackage{algorithm}
\usepackage{algpseudocode}

%Math library
\usepackage{amsmath} % for math
\usepackage{amsthm} % for theorems and proofs
\usepackage{amssymb}


% Definition for theorems, lemma
\newtheorem{thm}{Theorem}[section]
 \theoremstyle{definition}
 \newtheorem{dfn}{Definition}[section]
 \theoremstyle{remark}
 \newtheorem{note}{Note}[section]
 \theoremstyle{plain}
 \newtheorem{lem}[thm]{Lemma}

\begin{document}

\title{Problem Set 4}
\date{03/04/2017}
\author{Noella James}
\maketitle
collaborators: none\\

\section*{Problem 4-2: OCD-2}

\subsection*{A}
The container sizes are 1, 3, and 4.
Let L = 6.
With the greedy algorithm, the containers chosen will be $4, 1, and\; 1$.\\
With the optimal solution, the containers chosen will be 3 and 3. \\
Therefore, the greedy algorithm uses 3 containers while the optimal solution uses only 2. Therefore, the greedy algorithm does not provide the optimal number of containers.\\

\subsection*{B}
Let $n_1$ be the number of containers of capacity 1 gallon, $n_3$ be the number of containers of capacity 3 gallons, and $n_4$ be the number of containers of capacity 4 gallons. The solution is the minimize the value of $n_1 + n_3 + n_4$ such that $n_1 \times 1 + n_3 \times 3 + n_4 \times 4 = L$.\\
Assume that the capacities are $\{c_1\,,c_2\,,\ldots\,,c_n\}$ such that $c_1 < c_2 < \ldots < c_n$ for a given container $i$ with capacity $c_i$ and a capacity $L'$ to be filled, the possibilities are as follows:\\
Let $\mathcal{O}$ be the optimal solution.\\
1. $c_i > L'$ thus $OPT(i, L') = OPT(i-1, L')$\\
2. $c_i \leq L'$ and $\notin \mathcal{O}$, thus $OPT(i, L') = OPT(i-1, L')$\\
3. $c_i \leq L'$ and $\in \mathcal{O}$, thus $OPT(i, L') = 1 + OPT(i, L' - c_i)$\\
For case 3, we use $i$ instead of $i-1$ because the same size container may be used more than once.\\
The recurrence relationship can be stated as \\\\
if $c_i > L'$ \\\\
$OPT(i, L') = OPT(i-1, L')$\\\\
otherwise\\\\
$OPT(i, L') = min(OPT(i-1, L'), 1 + OPT(i, L' - c_i))$\\


\subsection*{B-Algorithm}
\begin{algorithm}
\caption{OCD-2}\label{Dynamic algorithm for OCD}
\begin{algorithmic}[1]
\Procedure{OCD}{$L$} \Comment{$L$ is the capacity}
        \State Array $M[0\ldots 3, 0\ldots L]$
        \State Initialize $M[0, l] = 0$ for each $l=0\,,1\,,\ldots\,,L$
        \State Array $C[0..3]$
        \State $C[0] \gets 0$
        \State $C[1] \gets 1$
        \State $C[2] \gets 3$
        \State $C[3] \gets 4$
        \State Array $Counter[1..3]$
        \State Initialize $Counter[i] \gets 0$ for each $i=1\,,2\,,3$
	\For{$i = 1\ldots 3$}
		\For{$l = 0\ldots L$}
		        \If {$c_i > l$}
			      \State $M[i,l] \gets M[i-1,l]$
		        \Else \Comment{ Find $min(M[i-1, l], 1 + M[i, l-C[i]]$}
		        	      \If {$M[i-1, l] <  1 + M[i, l-C[i]]$}
			          	\State $M[i,l] \gets M[i-1, l]$
			       \Else
			       		\State $M[i,l] \gets 1 + M[i, l-C[i]]$
					\State $Counter[i] \gets Counter[i] + 1$
			       \EndIf   		            
		        \EndIf
		\EndFor
	\EndFor
	\State \Return $M[3][L]$
\EndProcedure
\end{algorithmic}
\end{algorithm}


\subsection*{B-Complexity}
The complexity of this algorithm is the same as the Knapsack problem. Therefore, the complexity is $O(nL)$ where $n$ is the number of distinct container types and $L$ is the total capacity to be stored. However, in this specific problem, if $n$ is very small and equal to 3, thus the complexity will be $O(3L)$ or $O(L)$.

\subsection*{B-Correctness}
\begin{lem}
The algorithm $OCD(L, n c[n])$ correctly computes $OPT(i, l)$ for each $i = 1,2,\ldots,n$ and $l = 0, 1, \ldots,L$.
\end{lem}
\begin{proof}
By definition, $OPT(0,l) = 0$ for all $l = 0, 1, \ldots,L$. 
Now, take some $j > 0$, and suppose by way of induction that $OCD(j,l)$ correctly computes $OPT(i,l)$ for all $i < j$ and for all $l = 0, 1, \ldots,L$. . By the induction hypothesis, we know that $OCD(j-1, l) = OPT(j-1,l)$ and $OCD(j,l - c_j) = 1 + OPT(j, l - c_j))$.  Note that $OCD(j, l - c_j)$ gets computed before $ODC(j,l)$ and thus the value has been computed.  Hence, it follows that 
$OPT(j,l) = min( OPT(j-1,l), 1 + OPT(j, l - c_j)) = OCD(j,l)$
\end{proof}
\end{document}
