% Notes from Class 6

\documentclass{article}

%Algorithm library
\usepackage{algorithm}
\usepackage{algpseudocode}

%Math library
\usepackage{amsmath} % for math
\usepackage{amsthm} % for theorems and proofs
\usepackage{amssymb}

%Graph library
\usepackage{tikz}
\usetikzlibrary{graphs, graphdrawing}
\usetikzlibrary{arrows.meta,chains,decorations.pathmorphing,calc}
\usegdlibrary{trees, layered}

% Definition for theorems, lemma
\newtheorem{thm}{Theorem}[section]
 \theoremstyle{definition}
 \newtheorem{dfn}{Definition}[section]
 \theoremstyle{remark}
 \newtheorem{note}{Note}[section]
 \theoremstyle{plain}
 \newtheorem{lem}[thm]{Lemma}

\begin{document}

\section{Minimum Spanning Tree}

\textbf{Input:} Connected, undirected, weighted $G=(V,E)$\\
\textbf{Output:} A tree on edges that minimizes sum of weights while spanning all vertices

\begin{tikzpicture}
  [scale=.4,auto=left,every node/.style={circle,fill=blue!20}]
  \node (n1) at (0,0) {0};
  \node (n2) at (4,4) {1};
  \node (n3) at (4, -4)  {2};
  \node (n4) at (8,0)  {3};


 % \foreach \from/\to in {n1/n2,n1/n4,n1/n3,n2/n3,n3/n2,n4/n3,n2/n5,n3/n5,n4/n5,n5/n4}
 %   \draw (\from) edge[->]  (\to);
 
  \draw  (n1) -- (n2) node [above, draw=none, fill=none, align=center, midway, text width=.4mm]{$10$};
  \draw  (n2) -- (n3) node [above, draw=none, fill=none, align=center, midway, text width=.4mm]{$8$};
  \draw   [very thick, green]  (n2) -- (n4) node [above, draw=none, fill=none, align=center, midway, text width=.4mm]{$3$};
  \draw   [very thick, green]  (n3) -- (n4) node [above, draw=none, fill=none, align=center, midway, text width=.4mm]{$4$};
  \draw   [very thick, green]  (n3) -- (n1) node [above, draw=none, fill=none, align=center, midway, text width=.4mm]{$7$};  

\end{tikzpicture}\\\\


\subsection{Cycle Property}
\begin{lem}
Let $C$ be a cycle in $G$. Let $e \in C$ be a heaviest edge on $C$.  There, there is always an MST that does not contain $e$.
\end{lem}

\begin{proof}

Let $T$ be an MST where $e \in T$.\\
Consider $T - \{\,e\,\}$, where $e = (u,v)$\\
Let $A = \{ w \in V | w $ is reachable from $ u \in T - \{e\} \}$\\
There must be an edge $e'$ on $C$ that connected $A$ to $V-A$.\\
Consider $T' \triangleq T - \{ e \} \bigcup \{ e' \}$\\
\textsl{Claim 1:} Weight of $T'$ only smaller or equal to weight of $T$ because $e$ is the heaviest edge.\\
\textsl{Claim 2:} $T'$ spans all vertices.

\end{proof}


\section{Kruskal's Algorithm}

\begin{algorithm}
\caption{Kruskal's Algorithm}\label{Kruskal's Algorithm to find MST}
\begin{algorithmic}[1]
\Procedure{Kruskal's Algorithm}{$G$}\Comment{$G(V,E)$ weighted graph w/non-negative weights}
	\State $T \gets \{\}$
	\State Sort all edges according to weight. $e_1\, \leq e_2\, \leq \cdots \, \leq e_m$	
	\ForAll{ edges $e=(u,v)$ in order from to light to heavy}
		\If{$u$ not connected to $v$ }
			\State $T \gets T \bigcup \{e\}$
		\EndIf
	\EndFor
	\State \textbf{return} $T$ for $G$
\EndProcedure
\end{algorithmic}
\end{algorithm}

\begin{note}
You must pass on connected components, or else you will make a cycle.
\end{note}

\subsection{Complexity}
$n-1 \leq m \leq n^2$\\
$m = |E| \quad n = |V|$\\
$O (m \log n)$  
\begin{note}
The actual complexity is $O(m \log m)$ but it is aways better to write $\log n$ if it is a connected graph.
\end{note}

\section{Union-Find Data Structure}
Maintain a collection of disjoint sets. Ex: connected components of a graph.\\
Each disjoint set is represented by one of its members.\\

\begin{itemize}
	\item \textbf{Make-Set($X$):} Creates a new set with element $X$. $O(1)$
	\item \textbf{Find-Set($X$):} Returns the representative of the set $X$ belongs to. $O(\log n)$. depth of $X$ in its tree.
	\item \textbf{Union($X,Y$):} Merges the set represented by $X$ and $Y$. $O(1)$
\end{itemize}

Represent each set by a tree.  Root = representative.  Each element points to its parent in the tree.

Union by size: Whenever we union two sets, we make the root of the smaller tree, the child of the root of the larger tree.

\section{MST-KRUSKAL}
Implementation of Kruskal using Union-Find is described in Algorithm 2.

\begin{algorithm}
\caption{Kruskal's Algorithm}\label{Kruskal's Algorithm using Union-Find}
\begin{algorithmic}[1]
\Procedure{MST-KRUSKAL}{$G$}\Comment{$G(V,E)$ weighted graph w/non-negative weights}
	\State $T \gets \{\}$
	\ForAll{$v \in V$}
		\State Make-Set($v$)
	\EndFor
	\State Sort all edges according to weight. $e_1\, \leq e_2\, \leq \cdots \, \leq e_m$	
	\ForAll{ edges $e=(u,v) \in E$ in order from to light to heavy}
		\If{Find-Set($u$) \ne Find-Set($v$) }
			\State $T \gets T \bigcup \{e\}$
			\State Union($u$,$v$)
		\EndIf
	\EndFor
	\State \textbf{return} $T$ for $G$
\EndProcedure
\end{algorithmic}
\end{algorithm}

\section{Clustering}

\textbf{Input:} Point $P_1$ - $P_n$\\
\textbf{Distance Function:}$d (.\,,.)$\\
$\forall i\,,j$\\
$ d(P_i,P_j) = 0 \leftrightarrow i = j $\\
$ d(P_i, P_j) = d(P_j, P_i)$\\
$ d(P_i, P_j) \ge 0$\\
\textbf{Output:} Partition the points to $K$ sets where $K$ is a natural number.\\
So the spacing between clusters i s max.  Spacing = min distance between points in different clusters.

Ex: Netflix "you may also like"

Algorithm: Run Kruskal on graph 
$\{p_1\,, p_2\,, \ldots \,,p_n\}$ all edges $(p_i,p_j)$ each weighted with $d(p_i,p_j)$

Stop when you have reached $K$ connected components.

Let $D$ be the spacing Kruskal achieved. $D$ = weight of the $(k-1)^{th}$ heaviest edge in MST.\\
$\Rightarrow$ edges inside component are of weight $\leq D$

\end{document}