% Notes from Class 4

\documentclass{article}

%Algorithm library
\usepackage{algorithm}
\usepackage{algpseudocode}

%Math library
\usepackage{amsmath} % for math
\usepackage{amsthm} % for theorems and proofs
\usepackage{amssymb}

%Graph library
\usepackage{tikz}
\usetikzlibrary{graphs, graphdrawing}
\usetikzlibrary{arrows.meta,chains,decorations.pathmorphing,calc}
\usegdlibrary{trees, layered}

% Definition for theorems, lemma
\newtheorem{thm}{Theorem}[section]
 \theoremstyle{definition}
 \newtheorem{dfn}{Definition}[section]
 \theoremstyle{remark}
 \newtheorem{note}{Note}[section]
 \theoremstyle{plain}
 \newtheorem{lem}[thm]{Lemma}

\begin{document}

\section{Dijkstra's Algorithm}

Dijkstra's Algorithm is another example of greedy algorithms. \\\\

\begin{tikzpicture}
  [scale=.4,auto=left,every node/.style={circle,fill=blue!20}]
  \node (n1) {0};
  \node (n3) [right=of n1]  {2};
  \node (n2) [above=of n3] {1};
  \node (n4) [below=of n3]  {3};
  \node (n5) [right=of n3]  {4};

 % \foreach \from/\to in {n1/n2,n1/n4,n1/n3,n2/n3,n3/n2,n4/n3,n2/n5,n3/n5,n4/n5,n5/n4}
 %   \draw (\from) edge[->]  (\to);
 
  \draw  [->] (n1) -- (n2) node [above, draw=none, fill=none, align=center, midway, text width=.4mm]{$2$};
  \draw  [->] (n1) -- (n3) node [above, draw=none, fill=none, align=center, midway, text width=.4mm]{$8$};
  \draw  [->] (n1) -- (n4) node [above, draw=none, fill=none, align=center, midway, text width=.4mm]{$8$};
  
  \draw  [<->] (n2) -- (n3) node [above, draw=none, fill=none, align=center, midway, text width=.4mm]{$1$};

  \draw  [->] (n4) -- (n3) node [above, draw=none, fill=none, align=center, midway, text width=.4mm]{$1$};
  \draw  [->] (n2) -- (n5) node [above, draw=none, fill=none, align=center, midway, text width=.4mm]{$3$};
  \draw  [->] (n3) -- (n5) node [above, draw=none, fill=none, align=center, midway, text width=.4mm]{$1$};

  \draw  [<->] (n4) -- (n5) node [above, draw=none, fill=none, align=center, midway, text width=.4mm]{$2$};
 
\end{tikzpicture}\\\\

\textbf{\textit{Input :}}
Graph $G=(V,E)$ weighted w/non-negative weights.  $s \in V$ ($s$ is a vector in $G$).

\textbf{\textit{Output :}}
$\forall v \in V$ length of the shortest path from $s$ to $v$ ($\forall v $ in $V$)

\subsection{Algorithm}

Maintain $S \subseteq V$ : vertices for which we already have found shortest paths.
\textsl{Convention:}  If $(u,v) \notin E$ i.e its not an edge then $w(u,v) = \infty$

\begin{algorithm}
\caption{Dijkstra's Algorithm}\label{Dijkstra's Algorithm to find the shortest paths}
\begin{algorithmic}[1]
\Procedure{Dijkstra's Algorithm}{$G,s$}\Comment{$G(V,E)$ weighted graph w/non-negative weights, $s \in V,\; s$ is a vertex in $V$}
	\State $S \gets \{ s \}$\Comment{a set containing only s}
	\State $d(S) \gets 0$
	\Repeat
		\State Pick $v \in V - S$ that minimizes $min\,d(u) + w(u,v)$\Comment{$d \rightarrow $ distance, $w \rightarrow $ weight}
		\State $S \gets S \, \bigcup \, \{ v \}$\Comment{add $v$ to $S$}
		\State $d(v) \gets min\,d(u) + w(u,v)$
	\Until {There are no more vertices that can be considered.}
	\State \textbf{return} $\forall v \in V$ length of the shortest path from $s$ to $v$ ($\forall v $ in $V$).
	\EndProcedure
\end{algorithmic}
\end{algorithm}

\subsection{Correctness}

\begin{lem}
Every time a vertex $v \in V$ is added to $S$. $d(v)$ is the length of the shortest path from $s$ to $v$.
\end{lem}

\begin{proof} By Induction\\
\textbf{Base Case:} $S = \{ s \} \quad d(s) = 0$, indeed $0$ is the length of the shortest path $s  \rightsquigarrow s$\\
\textbf{Induction Hypotheses:} Lemma is $true$ for certain $s$.\\
\textbf{Induction Step:} We'll prove that the lemma is still $true$ after we add $v$ to $S$.\\
Let $u \in S$ such that $d(u) + w(u,v)$ is minimum.\\
There exists  a path $s  \rightsquigarrow v$ of length $d(u) + w(u,v)$ : the shortest path from $s$ to $u$, then the edge $(u,v)$\\
Assume as a way of contradiction that there is a path  $s  \rightsquigarrow v$ of length smaller the $d(u) + w(u,v)$. \\

$s  \in S \rightsquigarrow v \notin S$ \\
On path from $s$ to $v$ there much be an edge $(u',v') \in E$ where $u' \in S, v' \in S\;  \Rightarrow d(u') + w(u',v') < d(u) + w(u,v)$\\
$d(u') + w(u',v')$ is part of path that is shorter than $s  \rightsquigarrow u \rightarrow v$\\
So the algorithm would have picked $d(u') + w(u',v')$ or better and not $d(u) + w(u,v)$.\\
This is a contradiction.
\end{proof}

\begin{note}
If there are negative weights, $d(u') + w(u',v') < d(u) + w(u,v)$ does not hold.
\end{note}

\section{Priority Queue}
Maintain a set of elements, each has a value "key".\\

\begin{itemize}
	\item \textbf{Insert($Q\,,X$):} Insert $X$ with value key($X$)
	\item \textbf{Min($Q$):} Report if $Q \neq 0$ or $Q = 0$
	\item \textbf{Extract-Min($Q$):} Returns min and deletes it from $Q$
	\item \textbf{Decrease-Key($Q\,,X\,,k$):} Changes Key($X$) to $k$ if $k < $key($X$)
\end{itemize}

\section{Dijkstra's Algorithm Implementation}
$Q$ will maintain vertices, specifically $V - S$ (set minus)\\
key($v$) = min $d(u) + w(u,v)$ where $u \in S$\\\\

\begin{algorithm}
\caption{Dijkstra's Algorithm Implementation using Priority Queue}\label{Dijkstra's Algorithm to find the shortest paths}
\begin{algorithmic}[1]
\Procedure{Dijkstra's Algorithm Implementation}{$G,s$}\Comment{$G = V, E$  size: $m = |E| \; n = |V|$}
	\State $\forall v \in V\,, d(v) \gets \infty$
	\State $d(s) \gets 0$
	\State Insert all vertices to Q.\Comment{$O(n \log n$)}
	\While{$Q \neq 0$}	
		\State $u \gets $ Extract-min($Q$)\Comment{$O(n \log n)$}
		\ForAll{ neighbor $v$ of $u\,,\quad, v \in Q$}
			\State Decrease-Key ($Q\,,v\,,d(u)+w(u,v)$) 
		\EndFor \Comment{$O(m \log n)$}
	\EndWhile
	\EndProcedure
\end{algorithmic}
\end{algorithm}

\begin{tikzpicture}
  [scale=.4,auto=left,every node/.style={circle,fill=blue!20}]
  \node (n1) {0};
  \node (n3) [right=of n1]  {2};
  \node (n2) [above=of n3] {1};
  \node (n4) [below=of n3]  {3};
  \node (n5) [right=of n3]  {4};

 % \foreach \from/\to in {n1/n2,n1/n4,n1/n3,n2/n3,n3/n2,n4/n3,n2/n5,n3/n5,n4/n5,n5/n4}
 %   \draw (\from) edge[->]  (\to);
 
  \draw  [->] (n1) -- (n2) node [above, draw=none, fill=none, align=center, midway, text width=.4mm]{$2$};
  \draw  [->] (n1) -- (n3) node [above, draw=none, fill=none, align=center, midway, text width=.4mm]{$8$};
  \draw  [->] (n1) -- (n4) node [above, draw=none, fill=none, align=center, midway, text width=.4mm]{$8$};
  
  \draw  [<->] (n2) -- (n3) node [above, draw=none, fill=none, align=center, midway, text width=.4mm]{$1$};

  \draw  [->] (n4) -- (n3) node [above, draw=none, fill=none, align=center, midway, text width=.4mm]{$1$};
  \draw  [->] (n2) -- (n5) node [above, draw=none, fill=none, align=center, midway, text width=.4mm]{$3$};
  \draw  [->] (n3) -- (n5) node [above, draw=none, fill=none, align=center, midway, text width=.4mm]{$1$};

  \draw  [<->] (n4) -- (n5) node [above, draw=none, fill=none, align=center, midway, text width=.4mm]{$2$};
 
\end{tikzpicture}\\\\

\begin{equation*}
	\begin{bmatrix}
		0 & 1 & 2 & 3 & 4\\
		0 & \infty & \infty & \infty & \infty\\
		- & 2 & 8  & 8 & \infty\\
		- & - & 3  & -  & 5\\	
		- & - & -  & -  & 4\\
		- & - & -  & 6  & -\\
	\end{bmatrix}
\end{equation*}

Path: 0, 1, 2, 4, 3

\begin{note}
Always remember to add $d+w$
\end{note}

An implementation of priority queue is binary heap.  You can use an array to store all the values in a binary head, children "\ge" parent

\begin{itemize}
	\item \textbf{Insert($Q\,,X$):} $O (\log n)$
	\item \textbf{Min($Q$):} $O (1)$
	\item \textbf{Extract-Min($Q$):} $O (\log n)$
	\item \textbf{Decrease-Key($Q\,,X\,,k$):} $O (\log n)$
\end{itemize}

\begin{note}
Can sort with priority queues
n inserts + n extract-mins has to take $\Omega (n \log n)$
\end{note}

\end{document}