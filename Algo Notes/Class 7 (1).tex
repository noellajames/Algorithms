% Notes from Class 7 - 02/08/2017

\documentclass{article}

%Algorithm library
\usepackage{algorithm}
\usepackage{algpseudocode}

%Math library
\usepackage{amsmath} % for math
\usepackage{amsthm} % for theorems and proofs
\usepackage{amssymb}

% Definition for theorems, lemma
\newtheorem{thm}{Theorem}[section]
 \theoremstyle{definition}
 \newtheorem{dfn}{Definition}[section]
 \theoremstyle{remark}
 \newtheorem{note}{Note}[section]
 \theoremstyle{plain}
 \newtheorem{lem}[thm]{Lemma}

\begin{document}

\title{Divide and Conquer}
\date{02/07/2017}
\author{Noella James}
\maketitle

\section{Paradigm}

\begin{enumerate}
\item Break problem into sub-problems on smaller input size.
\item Solve each sub-problem recursively.
\item Combine solutions to overall solution.
\end{enumerate}

\section{Example: Merge Sort}

\framebox{7}
\framebox{8}
\framebox{1}
\framebox{5}
\framebox{12}
\framebox{3}
\framebox{6}
\framebox{9}

\begin{enumerate}
\item Divide the array into two halves
\item Sort each half
\item Merge the two halves
\end{enumerate}

\framebox{1}
\framebox{5}
\framebox{7}
\framebox{8}
\mbox{      }
\framebox{3}
\framebox{6}
\framebox{9}
\framebox{12}
 \\\\
\framebox{1}
\framebox{3}
\framebox{5}
\framebox{6}
\framebox{7}
\framebox{8}
\framebox{9}
\framebox{12}
\\\\

\section{Recurrence}

$T(n) =$ run-time on $n$ numbers.\\
\textbf{Recurrence:} $T(n) = 2 \times T(\frac{n}{2}) + \theta(n)$\\


\subsection{How to solve recurrences?}

Two methods for figuring out the solution.\\
One method for proving the solution once you already know it.

\textbf{Example:} \\
$T(n) = 2 T(\frac{n}{2}) + \theta (n)$\\
$T(1) = \theta (1) $

\textbf{First:} Work with explicit constants\\
$T(n) = 2 T(\frac{n}{2}) + cn$\\
$T(1) = c$

\subsection{First Method:} Recursion Tree

$T(n) \rightarrow cn$\\\\
$T(\frac{n}{2}) \qquad T(\frac{n}{2}) \qquad \rightarrow 2 \,c \,\frac{n}{2} = c\,n$\\\\
$T(\frac{n}{4}) \quad T(\frac{n}{4}) \quad T(\frac{n}{4}) \quad T(\frac{n}{4}) \qquad \rightarrow 4 \,c \,\frac{n}{4} = c\,n$\\\\

Overall depth of the tree = $\log n$\\\\
\begin{note}
every layer is $c\,n$
\end{note}

\begin{note}
Overall:  $c \, n \log n = \theta (n \log n)$ where $c$ is a constant.
\end{note}

\textbf{Why is it so important to work with explicit constants?}\\

\textsl{Example}\\\\
$T(n) = 2\,T(n-1) + \theta (1)$\\\\
$T(1) = \theta(1)$\\\\

$T(n) \rightarrow \theta(1) \rightarrow c$\\\\
$T(n-1) \qquad T(n-1) \rightarrow 2\,\theta(1) = \theta(1) \rightarrow 2\,c$\\\\
$T(n-2) \quad T(n-2) \quad T(n-2) \quad T(n-2) \rightarrow 4\,\theta(1) = \theta(1) \rightarrow 4\,c$\\\\

Assuming the total is $n \theta(1) = \theta(n)$ is WRONG.\\
$T(n) \ge 2^{(n-1)}$. \\  we have this issue because we didn't use explicit constants!

\subsection{Iteration Method}
Unroll the recurrence

$T(n) = 2\,T(n-1) + c$\\\\
$T(1) = c$\\\\

$T(n) = 2\,T(n-1) + c \\= 2(2\,T(n-2)+c) +c\\
	= 2(2(2\,T(n-3)+c)+c)+c\\
	=2(2(2 \ldots 2(T(1 +c) +c) + c) +\ldots ) + c$\\\\

Suppose that $T(1) = c$\\
$\therefore T(n) = 2^{n-1}\,c + 2^{n-2}\,c + 2^{n-3} \,c + \ldots + c$\\\\
* this is very subtle and delicate. you must be very careful *\\
$\therefore c ( 2^{n-1} + 2^{n-2} + 2^{n-3} + \ldots + 1) = c (2^n -1)$\\\\
$\therefore c\,(2^n -1) = \theta(2^n)$\\
\begin{note}
Geometric Sum: $a + a\,q + a\,q^2 + a\,q^2 + \ldots + a\,q^{n-1} = a\, \frac{q^n -1}{q-1}$
\end{note}

\subsubsection{Example}
$T(n) = 2\, T(\frac{n}{2}) + c\,n$\\
$T(1) = c$\\\\

$T(n) = 2 \, T(\frac{n}{2}) + c\,n = 2 (2 \, T(\frac{n}{4}) + c\,\frac{n}{2}) + c\,n\\
= 2^{\log n} T(1) + 2^{\log {n-1}} c\,n / 2^{\log {n-1}} + \ldots + 2 \, c\, \frac{n}{2} + c \, n\\
= n\,c\,(\log n + 1) = \theta(n \log n)$ 

\subsection{Substitution method: Proof by Induction}
\subsubsection{Example 1}
$T(n) = 2\, T(\frac{n}{2}) + c\,n$\\
$T(1) = c$\\\\

\textbf{Claim:} $T(n) = c\; n \; (\log n + 1) \quad \forall n$

\begin{proof} 
By induction on $n$\\
Base: $n = 1 \quad  T(1) = c = c \cdot (\log 1 + 1) = c$\\
Hypothesis: True for $n-1$\\
Induction or Step $n$:\\
$T(n) = 2\;T(\frac{n}{2}) + c\,n\\
= 2\,c\,\frac{n}{2} (\log{\frac{n}{2}}) + c\,n\\
= c\,n\,(\log n - 1 + 1) + c\,n\\
= c\,n\,\log n + c\,n$
\end{proof}

\begin{note}
$\\\log {\frac{a}{b}} = \log a - \log b$\\
$\log 2 = 1$
\end{note}

\subsubsection{Example 2}

$\\T(n) = 2\, T(n-1) + 1$\\
$T(1) = 1$\\\\

\textbf{Claim:} $T(n) = 2^n - 1 \quad \forall n$

\begin{proof} 
By induction on $n$\\
Base: $n = 1 \quad  T(1) = 1 = 2^1 - 1 = c$\\
Hypothesis: True for $n-1$\\
Induction or Step $n$:\\
$T(n) = 2\;T(n-1) + 1\\
= 2 (2^{n-1} -1) + 1\\
= 2^n - 2 + 1\\
= 2^n -1$
\end{proof}

\section{Master Theorem}
$T(1) = \theta(1) \qquad T(n) = a\, T(\frac{n}{b}) + \theta(n^k)\\
a \ge 1 \quad b \ge 1 \qquad a, b$ are constants. $k$ is a non-negative constant\\

Then\\
\begin{enumerate}
\item $k < \log_b a \quad \Rightarrow \quad T(n) = \theta(n^{\log_b a})$
\item $k = \log_b a \quad \Rightarrow \quad T(n) = \theta(n^{\log_b a} \log n)$
\item $k > \log_b a \quad \Rightarrow \quad T(n) = \theta(n^k)$
\end{enumerate}
	
	
\subsection{Example}	
$T(n) = a T(\frac{n}{b}) + \theta(n) \quad, a < b\\
T(n) = \theta(n)$	

\end{document}