\documentclass{article}

%Algorithm library
\usepackage{algorithm}
\usepackage{algpseudocode}

%Math library
\usepackage{amsmath} % for math
\usepackage{amsthm} % for theorems and proofs
\usepackage{amssymb}

%Graph library
\usepackage{tikz}
\usetikzlibrary{graphs, graphdrawing}
\usetikzlibrary{arrows.meta,chains,decorations.pathmorphing,calc}
\usegdlibrary{trees, layered}

% Definition for theorems, lemma
\newtheorem{thm}{Theorem}[section]
 \theoremstyle{definition}
 \newtheorem{dfn}{Definition}[section]
 \theoremstyle{remark}
 \newtheorem{note}{Note}[section]
 \theoremstyle{plain}
 \newtheorem{lem}[thm]{Lemma}

\begin{document}

\section{Breath First Search : BFS}

\subsection{Algorithm}

The BFS algorithm is detailed in Algorithm 1.  The output is $L_0\,,L_1\,,L_2\,,\dots L_n$ where $L_i\;=\;$vertices reachable from $s$ in $i$ steps.

\begin{algorithm}
\caption{Breath First Search}\label{Breath first search for graph $G$}
\begin{algorithmic}[1]
\Procedure{BFS}{$G(V,E),s$}\Comment{$G(V,E)$ is the input graph, $s\in V$}
	\State initialize array $mark$ of size $|V|$ with 0\Comment{complexity $O(|V|)$}
	\State $L_0 \gets \{s\}$\Comment{$L_0$ only contains $s$.  $O(1)$}
	\State $mark[s] \gets 1$\Comment{$O(1)$}
	\State $i \gets 1$\Comment{$O(1)$}
	\Repeat
		\State $L_i \gets$ unmarked neighbors of $L_{i-1}$\Comment{$O(1)$ for every edge. $\leq O|E|$}
		\State mark $L_i$ \Comment{Mark all the nodes in $L_i$.  $O(1)$ for every edge. $\leq O|E|$}
		\State $i \gets i-1$\Comment{Increment $i$.  $O(1)$ for every iteration. $\leq O|V|$}
	\Until{ $L_{i-1} = \phi$}
\EndProcedure\Comment{total complexity $O(|V|+|E|$}
\end{algorithmic}
\end{algorithm}

\subsection{Lemma}

\begin{note}
Lemma is a claim
\end{note}

\begin{lem}
$\forall\;i$ the vertices in $L_i$ are exactly the vertices at distance $i$ from $s$.
\end{lem}

\begin{proof}

\textbf{Proof By Induction:} On $i\;, L_i = \{$ vertices at distance $i$ from $s\}$\\

\textbf{Base Case:}\\
$i = 0.\quad L_0 = \{s\}\,,s$ only vertex at distance $0$ from $s$.\\

\textbf{Induction Hypotheses:}\\ 
$\forall\;i\leq k \quad L_i = \{$ vertices at distance $i$ from $s \}$\\

\textbf{Induction Step:}\\
For $k+1$ we want to show $L_{k+1} = \{$ vertices at distance $k+1$ from $s\}$\\
Suppose in $v \in L_{k+1} \; \exists \; u \in L_k \quad (u,v) \in E$ and $v$  is not marked\\

% drawing sigly figure
\begin{tikzpicture}
  [scale=.4,auto=left,every node/.style={circle,fill=gray!20}]
  \node (n0) {$s$};
  \node (n1) [right=of n0]  {$u$};
  \node (n2) [right=of n1] {$v$};
  
  \draw  [->,decorate,
    			decoration={snake, amplitude=.4mm, segment length=2mm, post length=1mm}] (n0) -- (n1)
			node [above, draw=none, fill=none, align=center, midway, text width=.4mm]{$k$};
  \draw  [->,decorate,
    			decoration={snake, amplitude=.4mm, segment length=2mm, post length=1mm}] (n1) -- (n2)
			node [above, draw=none, fill=none, align=center, midway, text width=.4mm]{$1$};
\end{tikzpicture}


inductive hypothesis: $u$ is at distance $k$ from $s$.\\
$\Rightarrow$ there exists a path of length $k+1$ from $s$ to $v$.
There is no path of length $i \le k + 1$ from $s$ to $v$, because $v$ is unmarked $\Rightarrow v \notin L_i \; $ for $ \; i \leq k$\\

Every vertex $v \in L_{k+1}$ is a neighbor of vertex $u \in L_k$.  Hence, there exists a length $k+1$ path from $s$ to $v$.\\

If there were a shorter path from $s \rightsquigarrow v$ then by hypothesis $v$ should have been in some layer $L_j$ for $j \le k$ and not in $L_{k+1}$

\end{proof}




\subsection{Example}

An example of running the BFS search on the graph with start vertex $0$.\\

\begin{tikzpicture}
  [scale=.8,auto=left,every node/.style={circle,fill=blue!20}]
  \node (n1) at (1,5) [label=left:$L_0$] {0};
  \node (n2) at (5,9) [label=$L_1$] {1};
  \node (n3) at (5,1) [label=below:$L_1$] {2};
  \node (n4) at (9,5) [label=right:$L_2$] {3};

  \foreach \from/\to in {n1/n2,n1/n3,n3/n2,n2/n4,n4/n3}
    \draw (\from) edge[->]  (\to);
 
\end{tikzpicture}


\subsection{Application}
Shortest path for unweighted graph.\\
Web crawling (eg. Google indexing).\\
Social networking (eg. people you might known)\\
Network broadcast.\\
Garbage collection in modern programming languages.\\
Model checking\\


\section{Depth First Search - DFS}

DFS is a back-tracking based search algorithm.  It is described as part of Algorithm 2.

\begin{algorithm}
\caption{Depth First Search}\label{Depth first search for graph $G$}
\begin{algorithmic}[1]
\Procedure{DFS}{$G(V,E),s$}\Comment{$G(V,E)$ is the input graph, $s\in V$}
	\State $mark[s] \gets 1$\Comment{Total for all vertices $\leq O|V|$}
	\For{each neighbor $v$ of $s$}\Comment{Total for all edges $\leq O|E|$}
		\If{$mark[v] \neq 1$}
			\State DFS($G\,,v$)
		\EndIf			
	\EndFor
\EndProcedure\Comment{total complexity $O(|V|+|E|)$}
\end{algorithmic}
\end{algorithm}

\begin{note}
Complexity\\
marking vertices $\leq O(|V|)$\\
going over neighborhoods $\leq O(|E|)$\\
Total: $O(|V|+|E|)$
\end{note}

\subsection{Example}

\begin{tikzpicture}
  [scale=.4,auto=left,every node/.style={circle,fill=blue!20}]
  \node (n1) at (1,5) {0};
  \node (n2) at (5,9) {1};
  \node (n3) at (5,1)  {2};
  \node (n4) at (9,5)  {3};

  \foreach \from/\to in {n1/n2,n1/n3,n3/n2,n2/n4,n4/n3}
    \draw (\from) edge[->]  (\to);
 
\end{tikzpicture}

One possible DFS starting 0:

\begin{tikzpicture}
  [scale=.4,auto=left,every node/.style={circle,fill=blue!20}]
  \node (n1) {0};
  \node (n2) [right=of n1] {1};
  \node (n4) [right=of n2]  {3};
  \node (n3) [right=of n4]  {2};

  \foreach \from/\to in {n1/n2,n2/n4,n4/n3}
    \draw (\from) edge[->]  (\to);
 
\end{tikzpicture}

Another possible DFS starting 0:

\begin{tikzpicture}
  [scale=.4,auto=left,every node/.style={circle,fill=blue!20}]
  \node (n1) {0};
  \node (n3) [right=of n1]  {2};
  \node (n2) [right=of n3] {1};
  \node (n4) [right=of n2]  {3};
 

  \foreach \from/\to in {n1/n3,n3/n2,n2/n4}
    \draw (\from) edge[->]  (\to);
 
\end{tikzpicture}


\subsection{Applications}
-cycle detection\\
-topological sort\\
-navigating mazes\\


\section{Greedy Algorithms}

\subsection{Interval Scheduling Problem}

Input: Activities \\
Output: Subset of non-overlapping activities that is as large as possible.\\

\begin{algorithm}
\caption{IntervalSchedule}\label{Interval Scheduling Based on a set of activities}
\begin{algorithmic}[1]
\Procedure{IntervalSchedule}{$A$}\Comment{$A = activities$}
	\Repeat
		\State Find activities with earliest finish time.
		\State Remove all overlapping activities.
	\Until {No Activities Remain}
	\State \textbf{return} Subset of non-overlapping activities that is as large as possible.
	\EndProcedure
\end{algorithmic}
\end{algorithm}


\begin{tikzpicture}
	\draw (1,1) -- (2,1) node [below, draw=none, fill=none, align=center, midway, text width=.4mm]{$1$};
	\draw (2.5,1) -- (4,1) node [below, draw=none, fill=none, align=center, midway, text width=.4mm]{$3$};
	\draw (4.5,1) -- (6,1) node [below, draw=none, fill=none, align=center, midway, text width=.4mm]{$6$};
	\draw (6.5,1) -- (8,1) node [below, draw=none, fill=none, align=center, midway, text width=.4mm]{$8$};

	\draw (1.5,2) -- (3.5,2) node [below, draw=none, fill=none, align=center, midway, text width=.4mm]{$2$};

	\draw (3,3) -- (5,3) node [below, draw=none, fill=none, align=center, midway, text width=.4mm]{$4$};
	\draw (5.25,3) -- (5.75,3) node [below, draw=none, fill=none, align=center, midway, text width=.4mm]{$5$};
	\draw (5.95,3) -- (7,3) node [below, draw=none, fill=none, align=center, midway, text width=.4mm]{$7$};

\end{tikzpicture}

The order, using algorithm 3 is: $1\,,3\,,5\,,7$ ($4$ activities)

\subsection{Proof that the algorithm is optimal}

\begin{lem}
At every step of the algorithm, there exists an optimal solution that picks all the activities that the algorithm picked.
\end{lem}

\begin{note}
The lemma implies that the algorithm is optimal.
\end{note}

\begin{proof}

\textbf{Proof By Induction:} On number of activities\\

\textbf{Base Case:}\\
$0$ activities. Obviously contained in optimal solution\\

\textbf{Induction Hypotheses:}\\ 
After algo picked $k$ activities, there is an optimized solution (OPT) that picks all k activities.\\

\textbf{Induction Step:}\\
After the algo picked $k+1$ activities.

\begin{tikzpicture}
 	[nonterminal/.style={
		% The shape:
		rectangle,
		% The size:
		minimum size=6mm,
		% The border:
		very thick, draw=red!50!black!50,
      		% The filling:
		top color=white,
		bottom color=black!20, % and something else at the bottom % Font
		font=\itshape
		}]
		\node [nonterminal] (n1) at (0,0) {algo $k$ activities};
 
		\draw (1.5,0) -- (3,0) node [above, draw=none, fill=none, align=center, midway]{activity}
			node [below, draw=none, fill=none, align=center, midway]{$(k+1)$};
		
		\node [nonterminal] (n2) at (5,0) {OPT $k$ activities};
		\draw (6.75,0) -- (8.25,0) node [above, draw=none, fill=none, align=center, midway]{activity}
			node [below, draw=none, fill=none, align=center, midway]{$(k+1)$};

		\draw
			($ (n1) + (1mm,0) $)
			-- ++(0,-1)
			-| ($ (n2) - (1mm,0) $) node [below, draw=none, fill=none, align=left, midway]{Inductive Hypothesis};

\end{tikzpicture}

By algo's design its $k+1$ activity has the earliest finish time.

\end{proof}

\subsection{Exchange Argument}
OPT' is just like OPT except instead of $(k+1)$th activity that OPT picks, OPT' picks the $(k+1)$th activity of algo\\
OPT' feasible \{no overlapping between activities\}\\
OPT' has as many activities as OPT and is therefore optimal.

\end{document}