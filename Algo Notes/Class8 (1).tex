% Notes from Class 8 - 02/08/2017

\documentclass{article}

%Algorithm library
\usepackage{algorithm}
\usepackage{algpseudocode}

%Math library
\usepackage{amsmath} % for math
\usepackage{amsthm} % for theorems and proofs
\usepackage{amssymb}

% Definition for theorems, lemma
\newtheorem{thm}{Theorem}[section]
 \theoremstyle{definition}
 \newtheorem{dfn}{Definition}[section]
 \theoremstyle{remark}
 \newtheorem{note}{Note}[section]
 \theoremstyle{plain}
 \newtheorem{lem}[thm]{Lemma}

\begin{document}

\title{Divide and Conquer - Part II}
\date{02/07/2017}
\author{Noella James}
\maketitle

\section{Integer Multiplication}

\textbf{Input:} Two n-bit numbers; $a\,,b$.\\
\textbf{Ouput:} $a \centerdot b$ in bit representation\\\\
\textbf{Example:} $1101 \times 0111 = 1011011$.  Simple algorithm with $O(n^2)$ bit operations.


\subsection{Divide and Conquer - First Attempt}

a \framebox{$a_1$}
\framebox{$a_0$}   $\qquad a = 2^{\frac{n}{2}} a_1 + a_0$
 \\\\
b
\framebox{$b_1$}
\framebox{$b_0$} $\qquad b = 2^{\frac{n}{2}} b_1 + b_0$
\\\\
\begin{equation*}
\begin{split}
a \centerdot b & = (2^{\frac{n}{2}} a_1 + a_0) (2^{\frac{n}{2}} b_1 + b_0)\\
& = 2^n a_1 b_1 + 2^{\frac{n}{2}} (a_1 b_0 + a_0 b_1) + a_0 b_0\\
\end{split}
\end{equation*}
Where $a_1 b_1\,, a_1 b_0\,, a_0 b_1\,, a_0 b_0$ are multiplication of $\frac{n}{2}$ bit numbers.\\\\
$a\centerdot b$  \framebox{$a_1 b_1$}
\framebox{$a_1 b_0 + a_0 b_1$}   
\framebox{$a_0 b_0$}\\\\
Overall,\\
\begin{equation*}
\begin{split}
T(n) &= 4 T(\frac{n}{2}) + O(n) \\
&= O(n^2)\\
\end{split}
\end{equation*}
Application of Masters theorem with $k=1\,,b=2\,,a=4\quad \log_b a = 2$\\

\subsection{Karatsuba Algorithm - 1960}
\begin{equation*}
\begin{split}
a\centerdot b & = 2^n a_1 b_1 + 2^{\frac{n}{2}}((a_1 + a_0)(b_1 +b_0) - a_1 b_1 - a_0 b_0) + a_0 b_0\\
(a_1 +  a_0) (b_1 + b_0) & = a_1 b_1 + a_1 b_0 + a_0  b_1 + a_0 b_0
\end{split}
\end{equation*}
Thus, we do 3 multiplications $a_1 b_1 \;, a_0 b_0 \;, (a_1 + a_0)(b_1 +b_0)$ rather than 4 \\

\begin{equation*}
\begin{split}
T(n)   & =   3 T(\frac{n}{2}) + O(n) \\
 & =  O(n^{log 3})\\
 & = O(n^{1.585})
\end{split}
\end{equation*}

\section{Matrix Multiplication}
\textbf{Input:} Two $n \times n$ matrices, $A$ and $B$.\\
\textbf{Output:} $A \centerdot B$

A \\
\framebox{$A_{11}$}
\framebox{$A_{12}$}    \\
\framebox{$A_{21}$}
\framebox{$A_{22}$}   \\\\
 
B \\
\framebox{$B_{11}$}
\framebox{$B_{12}$}    \\
\framebox{$B_{21}$}
\framebox{$B_{22}$}   \\\\
 
 $A \centerdot B = $\\
 \framebox{$C_{11}$}
\framebox{$C_{12}$}    \\
\framebox{$C_{21}$}
\framebox{$C_{22}$}   \\\\
The matrices by broken by $\frac{n}{2}$\\
The $C$ sub-parts can be computed as follows:\\
\begin{equation*}
\begin{split}
C_{11} & = A_{11} B_{11} + A_{12} B_{21}\\
C_{12} & = A_{11} B_{12} + A_{12} B_{22}\\
C_{21} & = A_{21} B_{11} + A_{22} B_{21}\\
C_{22} & = A_{21} B_{13} + A_{22} B_{22}\\
\end{split}
\end{equation*}

\subsection{First Attempt - Complexity}

\begin{equation*}
\begin{split}
T(n) & = 8 T(\frac{n}{2}) + O(n^2)\\
&= O(n^3)\\\\
\end{split}
\end{equation*}

\subsubsection{Applying Master's Theorem}
\begin{equation*}
\begin{split}
k &= 2\\
a & = 8\\
b &= 2\\
\log_2 8 & = 3\\
k & < 3\\
T(n) & = O(n^{\log_b a})\\
& = O(n^3)
\end{split}
\end{equation*}

\subsection{Strassen's Algorithm (1969)}
\subsubsection{Computation of M's}
\begin{equation*}
	\begin{split}
		M_1 &= (A_{11} + A_{22}) (B_{12} + B_{22})\\
		M_2 &= (A_{21}+ A_{22}) B_{11}\\
		M_3 &= A_{11} (B_{12} - B_{22})\\
		M_4 &= A_{22} (B_{21} - B_{11})\\
		M_5 &= (A_{11} + A_{12}) B_{22}\\
		M_6 &= (A_{21} + A_{11}) (B_{11} + B_{12})\\
		M_7 &= (A_{12} - A_{22})(B_{21} + B_{22})
	\end{split}
\end{equation*}

\subsubsection{Computation of C's}
\begin{equation*}
	\begin{split}
		C_{11} & = M_1 + M_4 - M_5 + M_7\\
		C_{12} & = M_3 + M_5\\
		C_{21} & = M_2 + M_4\\
		C_{22} & = M_1 - M_2 + M_3 + M_6
	\end{split}
\end{equation*}

\subsubsection{Complexity}
\begin{equation*}
	\begin{split}
		T(n) &= 7\, T(\frac{n}{2}) + \theta(n^2)\\
		& = O(n^{\log 7})\\
		& = O(n^{2.8})
	\end{split}
\end{equation*}
	
\end{document}