\documentclass{article}
\usepackage{algorithm}
\usepackage{algpseudocode}
\usepackage{amsmath} % for math
\usepackage{amsthm} % for theorems and proofs
\usepackage{amssymb}
\usepackage{tikz}
\usepackage{pgfplots}
\usetikzlibrary{graphdrawing}
\usetikzlibrary{graphs}
\usegdlibrary{trees}
\begin{document}



%Notes for class 2 of algorithm class

\section{Introduction}

How do you solve Rubik's cube\\
You use a graph\\
Graph Vertices = Configurations of the cube\\
Graph Edges = transitions between configurations.\\
The graph is undirected as it is possible to go back and forth between various configurations.\\

You would use a BFS search to find the solution to a cube as it is feasible to find solution to the puzzle is less than 20 rotations. Though, the number is possible configuration is $43,252,003,274,469,356,004$ the shortest path to the solved state is $\leq 20$ 

\section{Representation of graphs}

Graphs can be represented either as adjacency matrix or as adjacency lists

\subsection{Adjacency matrix}

A graph can be represented as a two dimensional array with the row and columns representing the vertexes and each element $[u,v]$ representing the presence of an edge from $u$ to $v$.\\
For an undirected graph, the value at index $[u,v]$ would be same as the value at $[v,u]$.\\

An example representation is shown below.
\begin{equation*}
	\begin{bmatrix}
		1 & 0 & 0\\
		0 & 1 & 1\\
		1 & 1 & 0 
	\end{bmatrix}
\end{equation*}
	
An adjacency matrix representation support the check if there is an edge from $u$ to $v$ in constant time.  However, to compute the number of degree of a vertex or to find all the edges for a vertex requires $O(n)$ time. Additionally, the space requirement for this representation is $O(n^2)$.\\




\subsection{Adjacency list}

In the adjacency list representation, the vertex's are represented as an array. Each element of the array contains a list of vertex's for which an edge exists from the vertex under-consideration. Thus index of $u$ contains the list of vertex's for which an edge exists from $u$.

\begin{equation*}
	\begin{bmatrix}
		0\\
		1\\
		2\\
		3
	\end{bmatrix}
	\begin{matrix}
		[1 \quad 2]\\
		[  ]\\
		[1 \quad 3]\\
		[0 \quad 1 \quad 2]
	\end{matrix}
\end{equation*}

An example representation of directed graph based on adjacency list is shown above. Vertex $V_0$ has edges to $V_1$ and  $V_2$.  Whereas Vertex $V_1$ does not have edges to any other node.  The vertex $V_3$ has edges to all the other three edges.

The overall storage requirement for adjacency list representation is $O(V + E)$.  The size of the array is $O(V)$ and the count of all the edges is $O(E)$.  Thus, the overall storage requirement is $O(V+E)$.

\section{Search}

The search on a graph can be defined in two ways

\subsection{vertices reachable from $s$}

Input:\\
 		$Graph G = (V,E)$ \\
		$Vertex\; s \in V$\\
\\
Output:\\
List of vertexes reachable from $s$\\


\subsection{path from $s$ to $t$}
Input:\\
 		$Graph G = (V,E)$ \\
		$Vertex\; s, t \in V$\\
		$where, s\leftarrow solved\;configuration, t\leftarrow initial\;configuration$
\\
Output:\\
Is there a patch from $s$ to $t$\\

\subsection{Generic Search Algorithm}

The basic search algorithm is shown in Algorithm 1.\\

\begin{algorithm}
\caption{Generic Search Algorithm}\label{A generic search algorithm}
\begin{algorithmic}[1]
\Procedure{GenericSearch}{$G(V,E),s$}\Comment{$s$ is the start vertex}
	\State $S\gets \{s\}$\Comment{Initialize $S$ with the start vertex $s$}
	\Repeat
		\State pick an edge $(x,y) \in E\,,where\; x\in S\,,y\notin S$
		\State $S\gets S\bigcup \{y\}$
	\Until{there are no more edges as above}
	\State \textbf{return} S\Comment{$S$ has list of all the vertexes reachable from $s$}
\EndProcedure
\end{algorithmic}
\end{algorithm}

You can use an array "mark" to mark vertices that are in $S$. This will allow us to find in $O(1)$ time whether a vertex is in $S$.  A modified version of the algorithm with the "mark" array is shown below. This is shown in Algorithm 2.\\

\begin{algorithm}
\caption{Generic Search Algorithm with marking}\label{A generic search algorithm}
\begin{algorithmic}[1]
\Procedure{GenericSearch}{$G(V,E),s$}\Comment{$s$ is the start vertex}
	\State initialize array $mark$ of size $|V|$ with 0\Comment{complexity $O(V)$}
	\State $S\gets \{s\}$\Comment{Initialize $S$ with the start vertex $s$}
	\State $mark[s] \gets 1$
	\Repeat
		\State pick an edge $(x,y) \in E\,,where\; x\in S\,,y\notin S$
		\State $S\gets S\bigcup \{y\}$
	\Until{there are no more edges as above}
	\State \textbf{return} S\Comment{$S$ has list of all the vertexes reachable from $s$}
\EndProcedure
\end{algorithmic}
\end{algorithm}



\newtheorem{thm}{Theorem}
\begin{thm}
Every vertex that is added to $S$ is reachable from $s$.
\end{thm}
\begin{proof}
We prove this by induction on the number of iterations.\\
\textbf{Base:}\\
 Assertion $S = \{s\}$\\
$S$ is reachable from $s$.\\
\textbf{Induction Hypotheses:}\\ 
True for $i$ iterations.\\
\textbf{Induction Step:}\\
Let us prove for iteration $i+1$.
Pick edge $(x,y \in E, x\in S\,,y\notin S)$.\\
add $y$.\\
We claim that $y$ is reachable from $s$.\\
\\
$x\in S$, so $x$ got added to $S$ before the $i+1$ iteration.\\
By induction hypotheses, $x$ is reachable from $s$.\\
Since $(x,y) \in E$, $y$ is reachable from $s$.\\
$s \rightsquigarrow x \rightsquigarrow y$

\end{proof}


\subsection{Complexity}
Note that there are at-most $|E|$ iterations since an edge $(x,y)$ never gets picked twice. However, the array initialization of the "mark"  takes $O(|V|)$.  Thus the complexity of the graph search algorithm is $O(|V|+|E|)$.

\subsection{Example}

Consider the graph shown below:\\

\begin{tikzpicture}
  [scale=.8,auto=left,every node/.style={circle,fill=blue!20}]
  \node (n1) at (1,5)  {0};
  \node (n2) at (5,9)  {1};
  \node (n3) at (5,5)  {2};
  \node (n4) at (5,1)  {3};
  \node (n5) at (9,5)  {4};
  \node (n6) at (1,1) {5};


  \foreach \from/\to in {n1/n4,n5/n1,n1/n2,n2/n5,n2/n3,n3/n4}
    \draw (\from) -- (\to);

\end{tikzpicture}

Assuming $s=0$, the set $S=\{0,1,2,3,4\}$, with vertex $5$ not reachable from vertex $0$.

\section{Connected Components}

\textbf{Connected Component:} A connected component is a set $S$ of vertexes of graph $G=(V,E)$ such that for every vertexes $x\in S\,,y\in S\;$ there exists a path from $x$ to $y$.\\

The term connected component is used for undirected graphs. The equivalent term for directed graph is strongly connected.

The connected components for a graph $G=(V,E)$ can be found by repeatedly applying the graph search algorithm.  The algorithm $3$ can be used to find the connected components of a graph $G$.  A more detailed algorithm is provided as part of algorithm $4$

\begin{algorithm}
\caption{Connected Components}\label{A algorithm to find the connected components of a graph $G$}
\begin{algorithmic}[1]
\Procedure{ConnectedComponents}{$G(V,E)$}\Comment{$G$ is the input graph}
	\Repeat
		\State pick a vertex $s \in V$\Comment{$v$ is not part of any connected component}
		\State invoke $GraphSearch(G,s)$ to find connected component for $s$.
		\State remove component from Graph $G$
	\Until{$G$ is empty}
\EndProcedure
\end{algorithmic}
\end{algorithm}


\begin{algorithm}
\caption{Connected Components - Detailed}\label{A algorithm to find the connected components of a graph $G$}
\begin{algorithmic}[1]
\Procedure{ConnectedComponents}{$G(V,E)$}\Comment{$G$ is the input graph}
	\State initialize array $connected$ of size $|V|$ with 0\Comment{complexity $O(V)$}
	\State $S'\gets \phi$\Comment{Initialize $S'$ as an empty set.}
	\Repeat
		\State pick a vertex $v$ from $connected$ where $connected[v] = 0$.\Comment{$v$ is not part of any connected component}
		\State $connected[v] \gets 1$
		\State $T\gets GraphSearch(G,v)$
		\ForAll{$t \in T$}
			\State $connected[t] \gets 1$
		\EndFor
		\State $S'\gets S'\bigcup T$
	\Until{all elements of $connected$ is set to $1$}
	\State \textbf{return} $S'$\Comment{$S'$ is a set of all connected components of $G$}
\EndProcedure
\end{algorithmic}
\end{algorithm}

\subsection{Complexity}

The algorithmic complexity of finding all connected graphs is $O(|V| + |E|)$.  I will provide the proof later.

\section{Breath First Search : BFS}

A Breath First Search (BFS) is a graph search algorithm such that for every vertex reachable from $s$, the search finds it in the smallest \# of steps.

The BFS algorithm is detailed in Algorithm 5.  The output is $L_0\,,L_1\,,L_2\,,\dots L_n$ where $L_i\;=\;$vertices reachable from$s$ in $i$ steps.

\begin{algorithm}
\caption{Breath First Search}\label{Breath first search for graph $G$}
\begin{algorithmic}[1]
\Procedure{BFS}{$G(V,E),s$}\Comment{$G(V,E)$ is the input graph, $s\in V$}
	\State initialize array $mark$ of size $|V|$ with 0\Comment{complexity $O(|V|)$}
	\State $L_0 \gets \{s\}$
	\State $mark[s] \gets 1$
	\State $i \gets 1$
	\Repeat
		\State add vertex $u$ to $L_i$ for which $mark[u] = 0$
		\State $mark[u] \gets 1$ 
		\State $i \gets i+1$
	\Until{ there is no vertex $v$ for which $mark[v] = 0$}\Comment{complexity $O(|E|)$}
\EndProcedure\Comment{total complexity $O(|V|+|E|$}
\end{algorithmic}
\end{algorithm}

\subsection{Example}

An example of running the BFS search on the graph with start vertex $0$.\\

\begin{tikzpicture}
  [scale=.8,auto=left,every node/.style={circle,fill=blue!20}]
  \node (n1) at (1,5) [label=left:$L_0$] {0};
  \node (n2) at (5,9) [label=$L_1$] {1};
  \node (n3) at (5,1) [label=below:$L_1$] {2};
  \node (n4) at (9,5) [label=right:$L_2$] {3};

  \foreach \from/\to in {n1/n2,n1/n3,n3/n2,n2/n4,n4/n3}
    \draw (\from) edge[->]  (\to);
 
\end{tikzpicture}

\end{document}




